%%%% For double-blind review submission, w/o CCS and ACM Reference (max submission space)
\documentclass[10pt, sigplan]{acmart}
%%\settopmatter{printfolios=true,printccs=false,printacmref=false}
%% For double-blind review submission, w/ CCS and ACM Reference
%\documentclass[sigplan,10pt,review,anonymous]{acmart}\settopmatter{printfolios=true}
%% For single-blind review submission, w/o CCS and ACM Reference (max submission space)
%\documentclass[sigplan,10pt,review]{acmart}\settopmatter{printfolios=true,printccs=false,printacmref=false}
%% For single-blind review submission, w/ CCS and ACM Reference
%\documentclass[sigplan,10pt,review]{acmart}\settopmatter{printfolios=true}
%% For final camera-ready submission, w/ required CCS and ACM Reference
%\documentclass[sigplan,10pt]{acmart}\settopmatter{}


%% Conference information
%% Supplied to authors by publisher for camera-ready submission;
%% use defaults for review submission.
%\acmConference[PL'17]{ACM SIGPLAN Conference on Programming Languages}{January 01--03, 2017}{New York, NY, USA}
%\acmYear{2017}
%\acmISBN{} % \acmISBN{978-x-xxxx-xxxx-x/YY/MM}
%\acmDOI{} % \acmDOI{10.1145/nnnnnnn.nnnnnnn}
%\startPage{1}

%% Copyright information
%% Supplied to authors (based on authors' rights management selection;
%% see authors.acm.org) by publisher for camera-ready submission;
%% use 'none' for review submission.
\setcopyright{none}
%\setcopyright{acmcopyright}
%\setcopyright{acmlicensed}
%\setcopyright{rightsretained}
%\copyrightyear{2017}           %% If different from \acmYear

%% Bibliography style

%\bibliographystyle{ACM-Reference-Format}

%% Citation style
%\citestyle{acmauthoryear}  %% For author/year citations
\citestyle{acmnumeric}     %% For numeric citations
%\setcitestyle{nosort}      %% With 'acmnumeric', to disable automatic
                            %% sorting of references within a single citation;
                            %% e.g., \cite{Smith99,Carpenter05,Baker12}
                            %% rendered as [14,5,2] rather than [2,5,14].
%\setcitesyle{nocompress}   %% With 'acmnumeric', to disable automatic
                            %% compression of sequential references within a
                            %% single citation;
                            %% e.g., \cite{Baker12,Baker14,Baker16}
                            %% rendered as [2,3,4] rather than [2-4].


%%%%%%%%%%%%%%%%%%%%%%%%%%%%%%%%%%%%%%%%%%%%%%%%%%%%%%%%%%%%%%%%%%%%%%
%% Note: Authors migrating a paper from traditional SIGPLAN
%% proceedings format to PACMPL format must update the
%% '\documentclass' and topmatter commands above; see
%% 'acmart-pacmpl-template.tex'.
%%%%%%%%%%%%%%%%%%%%%%%%%%%%%%%%%%%%%%%%%%%%%%%%%%%%%%%%%%%%%%%%%%%%%%


%% Some recommended packages.
\usepackage{booktabs}   %% For formal tables:
                        %% http://ctan.org/pkg/booktabs
\usepackage{subcaption} %% For complex figures with subfigures/subcaptions
                        %% http://ctan.org/pkg/subcaption
\usepackage{xspace}
\usepackage{graphicx}
\usepackage{ifthen}
\usepackage{pgfplots}
\usepackage{listings}
\usepackage{multirow}
\usepgfplotslibrary{statistics}
\usepackage{dblfloatfix} %enable fig at bottom of page
%



% Source Code
%\usepackage{color}
%\usepackage{textcomp}
%\usepackage{listings}
%\usepackage{ulem}
%\usepackage[T1]{fontenc}
%\usepackage{times}
% \usepackage{needspace}
 

% Source Code
\usepackage{color}
\usepackage{textcomp}
\usepackage{listings}

\definecolor{source}{gray}{0.85}% my comment style
\newcommand{\myCommentStyle}[1]{{\footnotesize\sffamily\color{gray!100!white} #1}}
%\newcommand{\myCommentStyle}[1]{{\footnotesize\sffamily\color{black!100!white} #1}}

% my string style
\newcommand{\myStringStyle}[1]{{\footnotesize\sffamily\color{violet!100!black} #1}}
%\newcommand{\myStringStyle}[1]{{\footnotesize\sffamily\color{black!100!black} #1}}

% my symbol style
\newcommand{\mySymbolStyle}[1]{{\footnotesize\sffamily\color{violet!100!black} #1}}
%\newcommand{\mySymbolStyle}[1]{{\footnotesize\sffamily\color{black!100!black} #1}}

% my keyword style
\newcommand{\myKeywordStyle}[1]{{\footnotesize\sffamily\color{green!70!black} #1}}
%\newcommand{\myKeywordStyle}[1]{{\footnotesize\sffamily\color{black!70!black} #1}}

% my global style
\newcommand{\myGlobalStyle}[1]{{\footnotesize\sffamily\color{blue!100!black} #1}}
%\newcommand{\myGlobalStyle}[1]{{\footnotesize\sffamily\color{black!100!black} #1}}

% my number style
\newcommand{\myNumberStyle}[1]{{\footnotesize\sffamily\color{brown!100!black} #1}}
%\newcommand{\myNumberStyle}[1]{{\footnotesize\sffamily\color{black!100!black} #1}}

\lstset{
language={},
% characters
tabsize=3,
escapechar={!},
keepspaces=true,
breaklines=true,
alsoletter={\#},
literate={\$}{{{\$}}}1,
breakautoindent=true,
columns=fullflexible,
showstringspaces=false,
% background
frame=single,
aboveskip=1em, % automatic space before
framerule=0pt,
basicstyle=\footnotesize\sffamily\color{black},
keywordstyle=\myKeywordStyle,% keyword style
commentstyle=\myCommentStyle,% comment style
frame=single,%
backgroundcolor=\color{source},
% numbering
stepnumber=1,
numbersep=10pt,
numberstyle=\tiny,
numberfirstline=true,
% caption
captionpos=b,
% formatting (html)
moredelim=[is][\bfseries]{<b>}{</b>},
moredelim=[is][\textit]{<i>}{</i>},
moredelim=[is][\underbar]{<u>}{</u>},
moredelim=[is][\color{red}\uwave]{<wave>}{</wave>},
moredelim=[is][\color{red}\sout]{<del>}{</del>},
moredelim=[is][\color{blue}\underbar]{<ins>}{</ins>},
% smalltalk stuff
morecomment=[s][\myCommentStyle]{"}{"},
%    morecomment=[s][\myvs]{|}{|},
morestring=[b][\myStringStyle]',
moredelim=[is][]{<sel>}{</sel>},
moredelim=[is][]{<rcv>}{</rcv>},
moredelim=[is][\itshape]{<symb>}{</symb>},
moredelim=[is][\scshape]{<class>}{</class>},
morekeywords={true,false,nil,self,super,thisContext},
identifierstyle=\idstyle,
}

\makeatletter
\newcommand*\idstyle[1]{%
\expandafter\id@style\the\lst@token{#1}\relax%
}
\def\id@style#1#2\relax{%
\ifnum\pdfstrcmp{#1}{\#}=0%
% this is a symbol
\mySymbolStyle{\the\lst@token}%
\else%
\edef\tempa{\uccode`#1}%
\edef\tempb{`#1}%
\ifnum\tempa=\tempb%
% this is a global
\myGlobalStyle{\the\lst@token}%
\else%
\the\lst@token%
\fi%
\fi%
}
\makeatother


%\newcommand{\ct}{\lstinline[backgroundcolor=\color{white}]}
%\newcommand{\needlines}[1]{\Needspace{#1\baselineskip}}
\newcommand{\lct}{\texttt}

\lstnewenvironment{code}{%
    \lstset{%
    % frame=lines,
    frame=single,
    framerule=0pt,
    mathescape=false
    }%
    \noindent%
    \minipage{\linewidth}%
}{%
    \endminipage%
}%


\lstnewenvironment{codeWithLineNumbers}{%
    \lstset{%
    % frame=lines,
    frame=single,
    framerule=0pt,
    mathescape=false,
    numbers=left
    }%
    \noindent%
    \minipage{\linewidth}%
}{%
    \endminipage%
}%



\newenvironment{codeNonSmalltalk}
{\begin{alltt}\sffamily}
{\end{alltt}\normalsize}



\usepackage{xcolor}
\newcommand{\todo}[1]{\color{orange}\fbox{\bfseries\sffamily\scriptsize TODO:}{\sf\small$\blacktriangleright$\textit{#1}$\blacktriangleleft$}\color{black}}
\newcommand{\sk}[1]{\color{blue}\fbox{\bfseries\sffamily\scriptsize Sophie:}{\sf\small$\blacktriangleright$\textit{#1}$\blacktriangleleft$}\color{black}}
\newcommand{\cba}[1]{\color{purple}\fbox{\bfseries\sffamily\scriptsize Clement:}{\sf\small$\blacktriangleright$\textit{#1}$\blacktriangleleft$}\color{black}}
%\newcommand*{rotatebox{75}}




% Source Code
%\usepackage{color}
%\usepackage{textcomp}
%\usepackage{listings}
%\usepackage{ulem}
%\usepackage[T1]{fontenc}
%\usepackage{times}
% \usepackage{needspace}
 

% Source Code
\usepackage{color}
\usepackage{textcomp}
\usepackage{listings}

\definecolor{source}{gray}{0.85}% my comment style
\newcommand{\myCommentStyle}[1]{{\footnotesize\sffamily\color{gray!100!white} #1}}
%\newcommand{\myCommentStyle}[1]{{\footnotesize\sffamily\color{black!100!white} #1}}

% my string style
\newcommand{\myStringStyle}[1]{{\footnotesize\sffamily\color{violet!100!black} #1}}
%\newcommand{\myStringStyle}[1]{{\footnotesize\sffamily\color{black!100!black} #1}}

% my symbol style
\newcommand{\mySymbolStyle}[1]{{\footnotesize\sffamily\color{violet!100!black} #1}}
%\newcommand{\mySymbolStyle}[1]{{\footnotesize\sffamily\color{black!100!black} #1}}

% my keyword style
\newcommand{\myKeywordStyle}[1]{{\footnotesize\sffamily\color{green!70!black} #1}}
%\newcommand{\myKeywordStyle}[1]{{\footnotesize\sffamily\color{black!70!black} #1}}

% my global style
\newcommand{\myGlobalStyle}[1]{{\footnotesize\sffamily\color{blue!100!black} #1}}
%\newcommand{\myGlobalStyle}[1]{{\footnotesize\sffamily\color{black!100!black} #1}}

% my number style
\newcommand{\myNumberStyle}[1]{{\footnotesize\sffamily\color{brown!100!black} #1}}
%\newcommand{\myNumberStyle}[1]{{\footnotesize\sffamily\color{black!100!black} #1}}

\lstset{
language={},
% characters
tabsize=3,
escapechar={!},
keepspaces=true,
breaklines=true,
alsoletter={\#},
literate={\$}{{{\$}}}1,
breakautoindent=true,
columns=fullflexible,
showstringspaces=false,
% background
frame=single,
aboveskip=1em, % automatic space before
framerule=0pt,
basicstyle=\footnotesize\sffamily\color{black},
keywordstyle=\myKeywordStyle,% keyword style
commentstyle=\myCommentStyle,% comment style
frame=single,%
backgroundcolor=\color{source},
% numbering
stepnumber=1,
numbersep=10pt,
numberstyle=\tiny,
numberfirstline=true,
% caption
captionpos=b,
% formatting (html)
moredelim=[is][\bfseries]{<b>}{</b>},
moredelim=[is][\textit]{<i>}{</i>},
moredelim=[is][\underbar]{<u>}{</u>},
moredelim=[is][\color{red}\uwave]{<wave>}{</wave>},
moredelim=[is][\color{red}\sout]{<del>}{</del>},
moredelim=[is][\color{blue}\underbar]{<ins>}{</ins>},
% smalltalk stuff
morecomment=[s][\myCommentStyle]{"}{"},
%    morecomment=[s][\myvs]{|}{|},
morestring=[b][\myStringStyle]',
moredelim=[is][]{<sel>}{</sel>},
moredelim=[is][]{<rcv>}{</rcv>},
moredelim=[is][\itshape]{<symb>}{</symb>},
moredelim=[is][\scshape]{<class>}{</class>},
morekeywords={true,false,nil,self,super,thisContext},
identifierstyle=\idstyle,
}

\makeatletter
\newcommand*\idstyle[1]{%
\expandafter\id@style\the\lst@token{#1}\relax%
}
\def\id@style#1#2\relax{%
\ifnum\pdfstrcmp{#1}{\#}=0%
% this is a symbol
\mySymbolStyle{\the\lst@token}%
\else%
\edef\tempa{\uccode`#1}%
\edef\tempb{`#1}%
\ifnum\tempa=\tempb%
% this is a global
\myGlobalStyle{\the\lst@token}%
\else%
\the\lst@token%
\fi%
\fi%
}
\makeatother


%\newcommand{\ct}{\lstinline[backgroundcolor=\color{white}]}
%\newcommand{\needlines}[1]{\Needspace{#1\baselineskip}}
\newcommand{\lct}{\texttt}

\lstnewenvironment{code}{%
    \lstset{%
    % frame=lines,
    frame=single,
    framerule=0pt,
    mathescape=false
    }%
    \noindent%
    \minipage{\linewidth}%
}{%
    \endminipage%
}%


\lstnewenvironment{codeWithLineNumbers}{%
    \lstset{%
    % frame=lines,
    frame=single,
    framerule=0pt,
    mathescape=false,
    numbers=left
    }%
    \noindent%
    \minipage{\linewidth}%
}{%
    \endminipage%
}%



\newenvironment{codeNonSmalltalk}
{\begin{alltt}\sffamily}
{\end{alltt}\normalsize}



\begin{document}

%% Title information
\title[Garbage Collection Evaluation Infrastructure]{Garbage Collection Evaluation Infrastructure for the Cog VM}

%% Author with single affiliation.
\author{Sophie Kaleba}
                                        %% can be repeated if necessary
\affiliation{
  %\position{Position1}
  \department{Master student}              %% \department is recommended
  \institution{Universit� de Lille 1}            %% \institution is required
 % \streetaddress{Street1 Address1}
  \city{Lille}
  %\state{France}
  %\postcode{Post-Code1}
  \country{France}                    %% \country is recommended
}
\email{sophie.kaleba@etudiant.univ-lille1.fr}          %% \email is recommended

%% Author with two affiliations and emails.
\author{Cl\'ement B\'era}
\affiliation{
  % \position{}
	\department{Software Languages Lab}              %% \department is recommended
	\institution{Vrije Universiteit Brussel}            %% \institution is required
	\city{Brussel}
  % \state{}
  % \postcode{}
	\country{Belgium}                    %% \country is recommended
}
\email{clement.bera@vub.ac.be}          %% \email is recommended

%% Author with two affiliations and emails.
\author{Eliot Miranda}
\affiliation{
   \position{Virtual Machine Architect}
	%\department{Virtual Machine architect}              %% \department is recommended
	\institution{Feenk}            %% \institution is required
	\city{San Francisco}
  % \state{}
  % \postcode{}
	\country{California}                    %% \country is recommended
}
\email{eliot.miranda@gmail.com}          %% \email is recommended

\begin{abstract}
One of the next step to improve the Cog virtual machine, the default virtual machine for multiple programming languages such Pharo, Squeak and Newspeak, is to decrease the garbage collection pause time. A benchmarking infrastructure and reference garbage collection algorithm implementations are required to evaluate the performance of a new algorithm and compare it. Cog features a Mark-Compact algorithm, used in production, and we introduced a Mark-Sweep algorithm a the two reference algorithms. Benchmarks are built using two different approaches. Firstly, we turned code from memory intensive deployed applications into benchmarks to simulate real-world applications. Secondly, we built a configurable benchmark which simulates an application with different heap properties to be able to stress specific aspects of the memory management. We then evaluated the two reference algorithms on the infrastructure built to have reference benchmark results.
\end{abstract}

\keywords{Benchmark, Garbage Collector, Virtual machine, Managed runtime}  %% \keywords are mandatory in final camera-ready submission

\maketitle

\section{Introduction}
\label{sec:intro}

- Trying to build a more advanced full GC algo, need to measure..
- ref impl to compare
- Try to build benchmarks, both from real app and to stress specific aspects.
- We end by applying the bench on reference impl to validate the behavior is as expected.


\section{Reference Implementations}
\label{sec:ref}

To evaluate new full GC algorithms, we need to compare it against existing algorithms. Two of the most common GC algorithms are Mark-Compact and Mark-Sweep. We describe here the current Mark-Compact implementation, which has been in production for the past few years, and an implementation of Mark-Sweep we introduced as a reference earlier this year. 

\subsection{Mark-Compact}

Cog's Mark-Compact algorithm, named \emph{SpurPlanningCompactor}, implements the classic planning compaction algorithm for Spur.  It uses the fact that there is room for a forwarding pointer in all objects to store the eventual position of an object in the first field. It therefore first locates a large free chunk, or eden or a memory segment, to use as the savedFirstFieldsSpace, which it uses to store the first fields of objects that will be compacted. It then makes at least three passes through the heap.

The first pass plans where live movable objects will go, copying their first field to the next slot in savedFirstFieldsSpace, and setting their forwarding pointer to point to their eventual location (see planCompactSavingForwarders). The second pass updates all pointers in live pointer objects to point to objects' final destinations, including the fields in savedFirstFieldsSpace (see updatePointers and updatePointersInMobileObjects). The third pass moves objects to their final positions, unmarking objects, and restoring saved first fields as it does so (see copyAndUnmark: and copyAndUnmarkMobileObjects). If the forwarding fields of live objects in the to-be-moved portion of the entire heap won't fit in savedFirstFieldsSpace, then additional passes can be made until the entire heap has been compacted.  When snapshotting multiple passes are made, but when doing a normal GC only one pass is made.

Each pass uses a three finger algorithm, a simple extension of the classic two finger algorithm with an extra finger used to identify the lowest pinned object between the to and from fingers.  Objects are moved down, starting at the first free object or chunk, provided that they fit below the lowest pinned object above the to finger.  When an object won't fit the to finger is moved above the pinned object and the third finger is reset to the next pinned object below the from finger, if any.

\subsection{Mark-Sweep}

Cog's Mark-Compact algorithm, named \emph{SpurSweeper}, is a sweep-only algorithm, setting the compactor to SpurSweeper effectively changes the fullGC to a mark-sweep non-moving algorithm. 

SpurSweeper is a reference implementation if one wants to evaluate GC performance and compare it to a Mark-Sweep. It's also the only non-moving GC available right now which can be convenient for some experiments. One of the main reason why it was implemented is because advanced compaction algorithm includes a sweep phase (See SelectiveCompactor for example) and SpurSweeper allows to debug the sweep phase separately.

\subsection{Snapshot discussion}
Problem with the snapshot. and non compacting algo. Mention G1 and partial compaction too.

=> HybridCompactor solution.

\section{Building Macro-benchmarks}
\label{sec:bench}

2 directions. One is to build real benchs from real apps. Same direction as Dacapo... To be ported ?

\subsection{Moose benchmark}

Explain all the phases

> Growing heap:
[GCMooseBench new benchLoadFromMSE: b wildfly].
benchLoadFromMSE: fileName
mooseModel := MooseModel new importFromMSEStream: (StandardFileStream readOnlyFileNamed: fileName)

[(GCMooseBench withMooseModel: model) benchExpandProperties]. "optional, increases drastically the model size"
benchExpandProperties
	allProperties :=  mooseModel entityStorage collect: [ :anEntity | 
    		anEntity mooseInterestingEntity complexPropertyPragmas collect: [ :aPragma |
        		[ (anEntity mooseInterestingEntity perform: aPragma methodSelector) ] on: Error do: ['error'] ] ].

> Stable heap:
[(GCMooseBench withMooseModel: model) benchAnalysisPyramid].
benchAnalysisPyramid
	mooseModel overviewPyramidMetrics.
[(GCMooseBench withMooseModel: model) benchAnalysisSymbol].
benchAnalysisSymbol
	mooseModel allMethods symbolsUsedInName
[(GCMooseBench withMooseModel: model) benchAnalysisDeprecation].
benchAnalysisDeprecation
	mooseModel allModelClasses select: [ :each |
    		(each isAnnotatedWith: 'Deprecated') and: [ each clientTypes notEmpty ] ].

> Shrinking heap:
[(GCMooseBench withMooseModel: model) benchRelease].
benchRelease
	mooseModel := nil.
	allProperties := nil.
	3 timesRepeat: [Smalltalk garbageCollect].

\subsection{Configurable benchmark}

Sophie's section

\section{Experimental results}
\label{sec:valid}

It would be nice to evaluate both Sweeper and Planning on what we've just discussed.

\section{Related and Future Work}

\paragraph{GC benchmark suite.}
Dacapo + see ref in Dacapo.

\paragraph{ACDC.}
ACDC-js (Sophie you write a paragraph on that.

\paragraph{Future work: more benchmarks.}
need to add more benchmarks based on Dacapo ?

\paragraph{Future work: stress GC tests.}
stress cases from Configurable

%% Bibliography
\bibliographystyle{alpha}
\bibliography{sista}


\end{document}
